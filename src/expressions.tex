

\setcounter{chapter}{7}
\rSec0[expr]{Expressions}

Modify paragraph 8 to include a reference to \grammarterm{requires-expression}{}s.

\begin{quote}
\pnum
In some contexts, \defn{unevaluated operands} appear
(\added{\ref{expr.prim.req}, }\cxxref{expr.typeid}, \cxxref{expr.sizeof}, 
\cxxref{expr.unary.noexcept}, \cxxref{dcl.type.simple}).
\end{quote}

%%
%% Primary Expressions
%%
\rSec1[expr.prim]{Primary expressions}

In this section, add the
\grammarterm{requires-expression} to the rule for 
\grammarterm{primary-expression}.

\begin{quote}
\begin{bnf}
\nontermdef{primary-expression}\br
    literal\br
    \terminal{this}\br
    \terminal{(} expression \terminal{)}\br
    id-expression\br
    lambda-expression\br
    fold-expression\br
    \added{requires-expression}
\end{bnf}
\end{quote}

\setcounter{subsection}{3}
\rSec2[expr.prim.id]{Names}

Add new paragraphs to the end of this 
section.

\setcounter{Paras}{2}

\begin{quote}
\begin{addedblock}
\pnum
An \grammarterm{id-expression} that denotes the specialization of a concept
(\ref{temp.concept}) results in a prvalue of type \tcode{bool}. 
%
The expression is \tcode{true} if the concept's normalized 
(\ref{temp.constr.decl}) \emph{constraint-expression} is satisfied according
to the rules in \ref{temp.constr.constr} and \tcode{false} otherwise.
\enterexample
\begin{codeblock}
template<typename T> concept C = true;
static_assert(C<int>); // OK
\end{codeblock}
\exitexample
\enternote
A concept's constraints are also considered when using a template name
(\ref{temp.names}) and overload resolution (\ref{over}), and they are
compared during the the partial ordering of constraints 
(\ref{temp.constr.order}).
\exitnote

\end{addedblock}
\end{quote}

\begin{quote}
\begin{addedblock}
\pnum
A program that refers explicitly or implicitly to a function with associated 
constraints that are not satisfied (\ref{temp.constr.decl}), other than to 
declare it, is ill-formed.
% 
\enterexample
\begin{codeblock}
void f(int) requires false;

f(0);                      // error: cannot call \tcode{f}
void (*p1)(int) = f;       // error: cannot take the address of \tcode{f}
decltype(f)* p2 = nullptr; // error: the type \tcode{decltype(f)} is invalid
\end{codeblock}
In each case the associated constraints of \tcode{f} are not satisfied. In the 
declaration of \tcode{p2}, those constraints are required to be satisfied even 
though \tcode{f} is an unevaluated operand (Clause~\ref{expr}).
\exitexample
\end{addedblock}
\end{quote}


\setcounter{subsubsection}{1}
\rSec3[expr.prim.id.qual]{Qualified names}

Add \tcode{auto} to the \grammarterm{nested-name-specifier} grammar.

\begin{quote}
\begin{bnf}
\nontermdef{nested-name-specifier}\br
    \terminal{::}\br
    type-name \terminal{::}\br
    namespace-name \terminal{::}\br
    decltype-specifier \terminal{::}\br
    \added{\terminal{auto} \terminal{::}}\br
    nested-name-specifier identifier \terminal{::}\br
    nested-name-specifier \terminal{template}\opt \terminal{::}
\end{bnf}
\end{quote}

Add a new paragraph at the end of this section.

\begin{quote}
\begin{addedblock}
\setcounter{Paras}{5}
\pnum
In a \grammarterm{nested-name-specifier} of the form \tcode{auto::},
that \grammarterm{nested-name-specifier} designates a placeholder that 
will be replaced later according to the rules for placeholder deduction in
\ref{dcl.spec.auto}.
%
The replacement type deduced for a placeholder shall be a class or
enumeration type.
% 
\enterexample
\begin{codeblock}
template<typename T> concept C = sizeof(T) == sizeof(int);
template<int N> concept D = true;

struct S1 { int n; };
struct S2 { char c; };
struct S3 { struct X { using Y = int; }; };

int auto::* p1 = &S1::n; // \tcode{auto} deduced as \tcode{S1}

auto f = [](typename auto::X::Y) {};
f(S1()); // error: \tcode{auto} cannot be deduced from \tcode{S1()}
f.operator()<S3>(0); // OK

enum E { e };
auto g = [](decltype(auto::e)) {};
g.operator()<E>(e);  // OK
\end{codeblock}
In the declaration of \tcode{f}, the placeholder appears in a non-deduced 
context (\cxxref{temp.deduct.type}). It may be replaced later through the
explicit specification of template arguments.
\exitexample
\end{addedblock}
\end{quote}


%%
%% Requires expressions
%%
\setcounter{subsection}{6}
\rSec2[expr.prim.req]{Requires expressions}

Add this section to 8.1.

\begin{quote}
\begin{addedblock}

\pnum
A \grammarterm{requires-expression} provides a concise way to express 
requirements on template arguments. 
% 
A requirement is one that can be checked by name lookup
(\cxxref{basic.lookup}) or by checking properties of types and expressions.

\begin{bnf}
\nontermdef{requires-expression}\br
    \terminal{requires} requirement-parameter-list\opt requirement-body

\nontermdef{requirement-parameter-list}\br
    \terminal{(} parameter-declaration-clause\opt~\terminal{)}
  
\nontermdef{requirement-body}\br
    \terminal{\{} requirement-seq \terminal{\}}

\nontermdef{requirement-seq}\br
    requirement\br
    requirement-seq requirement

\nontermdef{requirement}\br
    simple-requirement\br
    type-requirement\br
    compound-requirement\br
    nested-requirement
\end{bnf}

\pnum
A \grammarterm{requires-expression} is a prvalue of type \tcode{bool} whose
value is described below.
Expressions appearing within a \grammarterm{requirement-body}
are unevaluated operands (Clause~\ref{expr}).

\pnum
\enterexample
A common use of \grammarterm{requires-expression}{}s is to define
requirements in concepts such as the one below:
\begin{codeblock}
template<typename T>
  concept R = requires (T i) {
    typename T::type;
    {*i} -> const typename T::type&;
  };
\end{codeblock}
A \grammarterm{requires-expression} can also be used in a 
\grammarterm{requires-clause} (\ref{temp}) as a way of writing ad hoc 
constraints on template arguments such as the one below:
\begin{codeblock}
template<typename T>
  requires requires (T x) { x + x; }
    T add(T a, T b) { return a + b; }
\end{codeblock}
The first \tcode{requires} introduces the 
\grammarterm{requires-clause}, and the second
introduces the \grammarterm{requires-expression}.
\exitexample
\enternote
Such requirements can also be written by defining them within
a concept.
\begin{codeblock}
template<typename T>
  concept C = requires (T x) { x + x; };

template<typename T> requires C<T> 
  T add(T a, T b) { return a + b; }
\end{codeblock}
\exitnote

\pnum
A \grammarterm{requires-expression} may introduce local parameters using a
\grammarterm{parameter-declaration-clause}
(\ref{dcl.fct}). 
%
A local parameter of a \grammarterm{requires-expression} shall not have a 
default argument.
%
Each name introduced by a local parameter is in scope from the point
of its declaration until the closing brace of the
\grammarterm{requirement-body}.
%
These parameters have no linkage, storage, or lifetime; they are only used
as notation for the purpose of defining \grammarterm{requirement}{}s.
%
The \grammarterm{parameter-declaration-clause} of a 
\grammarterm{requirement-parameter-list} shall
not terminate with an ellipsis.
\enterexample
\begin{codeblock}
template<typename T>
  concept C = requires(T t, ...) { // error: terminates with an ellipsis
    t; 
  };
\end{codeblock}
\exitexample

\pnum
The \grammarterm{requirement-body} is comprised of 
a sequence of \grammarterm{requirement}{}s.
%
These \grammarterm{requirement}{}s may refer to local 
parameters, template parameters, and any other declarations visible from the 
enclosing context. 

\pnum
The substitution of template arguments into a \grammarterm{requires-expression} 
may result in the formation of invalid types or expressions in its
requirements or the violation of the semantic constraints of those requirements.
In such cases, the \grammarterm{requires-expression} evaluates to \tcode{false};
it does not cause the program to be ill-formed.
The substitution and semantic constraint checking
proceeds in lexical order and stops when a condition that
determines the result of the \grammarterm{requires-expression} is encountered.
If no substitution is performed or substitution succeeds,
the \grammarterm{requires-expression} evaluates to \tcode{true}.
\enternote
If a \grammarterm{requires-expression} contains invalid types or expressions in
its requirements, and it does not appear within the declaration of a templated
entity, then the program is ill-formed.
\exitnote
If the substitution of template arguments into a \grammarterm{requirement} 
would always result in a substitution failure, the program is ill-formed; 
no diagnostic required.
%
\enterexample
\begin{codeblock}
template<typename T> concept C =
  requires {
    new int[-(int)sizeof(T)]; // ill-formed, no diagnostic required
  };
\end{codeblock}
\exitexample


%%
%% Simple requirements
%%
\rSec3[expr.prim.req.simple]{Simple requirements}

\begin{bnf}
\nontermdef{simple-requirement}\br
    expression \terminal{;}
\end{bnf}

\pnum
A simple requirement asserts the validity of an expression.
\enternote
The enclosing \grammarterm{requires-expression} will evaluate to \tcode{false}
if substitution of template arguments into the \grammarterm{expression} fails.
The \grammarterm{expression} is an unevaluated operand (Clause \ref{expr}).
\exitnote

\enterexample
\begin{codeblock}
template<typename T> concept C =
  requires (T a, T b) {
    a + b;  // \tcode{C<T>} is \tcode{true} if \tcode{a + b} is a valid expression
  };
\end{codeblock}
\exitexample


%%
%% Type requirements
%%
\rSec3[expr.prim.req.type]{Type requirements}

%%% This is not the same as typename-specifier because 'template' is neither
%%% required nor permitted in this context.
%%% FIXME: We should say that 'template' is not required in this context when
%%% we describe the rules for 'template'.
\begin{bnf}
\nontermdef{type-requirement}\br
    \terminal{typename} nested-name-specifier\opt type-name \terminal{;}
\end{bnf}

\pnum
A type requirement asserts the validity of a type.
\enternote
The enclosing \grammarterm{requires-expression} will evaluate to \tcode{false}
if substitution of template arguments fails.
\exitnote
\enterexample
\begin{codeblock}
template<typename T, typename T::type = 0> struct S;
template<typename T> using Ref = T&;

template<typename T> concept C =
  requires () {
    typename T::inner; // required nested member name
    typename S<T>;     // required class template specialization
    typename Ref<T>;   // required alias template substitution, fails if \tcode{T} is void
  };
\end{codeblock}
\exitexample

\pnum
A type requirement that names a class template specialization 
does not require that type to be complete 
(\cxxref{basic.types}).


%%
%% Compound requirements
%%
\rSec3[expr.prim.req.compound]{Compound requirements}
      
\begin{bnf}
\nontermdef{compound-requirement}\br
  \terminal{\{} expression \terminal{\}} \terminal{noexcept}\opt 
    return-type-requirement\opt~\terminal{;}

\nontermdef{return-type-requirement}\br
    trailing-return-type\br
    \terminal{->} constrained-parameter cv-qualifier-seq\opt~abstract-declarator\opt
\end{bnf}

\pnum
A \grammarterm{compound-requirement} asserts properties
of the \grammarterm{expression} \tcode{E}. After any template argument
substitution into the \grammarterm{compound-requirement},
the following properties are verified in this order:
%
\begin{itemize}
\item if the \tcode{noexcept} specifier is present,
\tcode{E} shall not be a potentially-throwing expression~(\cxxref{except.spec});

\item if the \grammarterm{return-type-requirement} is present, additional
requirements apply.

\begin{itemize}
\item If the \grammarterm{return-type-requirement} is a 
\grammarterm{trailing-return-type}, \tcode{E} shall be implicitly 
convertible to the type named by the
\grammarterm{trailing-return-type}.

\item If the \grammarterm{return-type-requirement} starts with a 
\grammarterm{constrained-parameter} (\ref{temp.param}), the 
\grammarterm{expression} is deduced against an invented function template
\tcode{F} using the rules in \cxxref{temp.deduct.call}. \tcode{F} is a
\tcode{void} function template with a single type template parameter
\tcode{T} declared with the \grammarterm{constrained-parameter}. 
It has a single \grammarterm{parameter} whose \grammarterm{type-specifier} is
\grammarterm{cv}-\tcode{T} followed by the \grammarterm{abstract-declarator}. 
\end{itemize}
\end{itemize}

%
\enterexample
\begin{codeblock}
template<typename T> concept C1 =
  requires(T x) {
    {x++};
  };
\end{codeblock}
The \grammarterm{compound-requirement} in \tcode{(C1)} 
requires that the expression \tcode{x++} is valid.
It is equivalent to a \grammarterm{simple-requirement}
with the same \grammarterm{expression}.

\begin{codeblock}
template<typename T> concept C2 =
  requires(T x) {
    {*x} -> typename T::inner;
  };
\end{codeblock}

The \grammarterm{compound-requirement} in \tcode{C2} 
requires that \tcode{*x} is a valid expression,
that \tcode{typename T::inner} is a valid type, and
that \tcode{*x} is implicitly convertible to
\tcode{typename T::inner}.

\begin{codeblock}
template<typename T, typename U> concept C3 = false;
template<typename T> concept C4 =
  requires(T x) {
    {*x} -> C3<U> const&;
  };
\end{codeblock}
The \grammarterm{compound-requirement} requires that \tcode{*x} be deduced
as an argument for the invented function:
\begin{codeblock}
template<C3<U> T> void f(U const&);
\end{codeblock}
In this case, deduction always fails since \tcode{C3} is \tcode{false}.

\begin{codeblock}
template<typename T> concept C5 =
  requires(T x) {
    {g(x)} noexcept;
  };
\end{codeblock}

The \grammarterm{compound-requirement} in \tcode{C5} 
requires that \tcode{g(x)} is a valid expression and
that \tcode{g(x)} is non-throwing.
\exitexample


%%
%% Nested requirements
%%
\rSec3[expr.prim.req.nested]{Nested requirements}

\begin{bnf}
\nontermdef{nested-requirement}\br
    requires-clause \terminal{;}
  \end{bnf}

\pnum
A \grammarterm{nested-requirement} can be used
to specify additional constraints in terms of local parameters.
The \grammarterm{constraint-expression} of the predicate constraint 
shall be satisfied.
% 
\enterexample
\begin{codeblock}
template<typename U> concept C = sizeof(U) == 1;

template<typename T> concept D =
  requires (T t) {
    requires C<decltype (+t)>;
  };
\end{codeblock}
\tcode{D<T>} is satisfied if \tcode{sizeof(decltype (+t)) == 1} (\ref{temp.constr.atomic}).
\exitexample
%
\enternote
Normalization of constraints appends a separate constraint for each
\grammarterm{nested-requirement} within a \grammarterm{requires-expression}
for the purpose of determining partial ordering (\ref{temp.constr.order}).
\exitnote
%
A local parameter shall not appear as an evaluated operand 
(\ref{expr}) of the \grammarterm{constraint-expression}.
\enterexample
\begin{codeblock}
template<typename T> 
  concept C = requires (T a) {
    requires sizeof(a) == 4; // OK
    requires a == 0;         // error: evaluation of a constraint variable
  }
\end{codeblock}
\exitexample






\begin{comment}

%%
%% Argument deduction constraints
%%
\rSec3[temp.constr.deduct]{Argument deduction constraints}

\pnum
An \defn{argument deduction constraint} is a constraint that specifies 
a requirement that the type of an \grammarterm{expression} \tcode{E}
can be deduced from a type \tcode{T}, when \tcode{T} includes one or more 
placeholders (\ref{dcl.spec.auto}).
% 
\enternote
An argument deduction constraint is introduced by a
\grammarterm{compound-requirement} (\ref{expr.prim.req.compound}) having a
\grammarterm{trailing-return-type} that contains one or more placeholders.
% 
In such a constraint, \tcode{E} is the \grammarterm{expression} of the 
\grammarterm{compound-requirement}, and \tcode{T} is the type specified
by the \grammarterm{trailing-return-type}.
\exitnote

\pnum
To determine if an argument deduction constraint is satisfied, invent
an abbreviated function template \tcode{f} with one parameter whose
type is \tcode{T} (\ref{dcl.fct}). 
% 
The constraint is satisfied if the resolution of the function call 
\tcode{f(E)} succeeds (\ref{over.match}).
% 
\enternote
Overload resolution succeeds when values are deduced for all invented
template parameters in \tcode{f} that correspond to the placeholders in 
\tcode{T}, and the constraints associated by any 
\grammarterm{constrained-type-specifier}{s} are satisfied.
\exitnote
% 
\enterexample
\begin{codeblock}
template<typename T, typename U> struct Pair;

template<typename T> concept C1 = true;

template<typename T> concept C2 = requires(T t) { {*t} -> Pair<C1&, auto>; }; }

template<C2 T> void g(T);

g((int*)nullptr); // error: constraints not satisfied.
\end{codeblock}
The invented abbreviated function template \tcode{f} for the 
\grammarterm{compound-requirement} in \tcode{C2} is:
\begin{codeblock}
void f(Pair<C1&, auto>);
\end{codeblock}
In the call \tcode{g((int*)nullptr)}, the constraints are not satisfied 
because no values can be deduced for the placeholders \tcode{C1} and 
\tcode{auto} from the expression \tcode{*t} when \tcode{t} has type
``pointer-to-\tcode{int}''.
\exitexample

\pnum
An argument deduction constraint $A$ is equivalent to another 
argument deduction constraint $B$ if and only if the 
\grammarterm{expression}{s} of $A$ and $B$ are equivalent
using the rules in \ref{temp.over.link} to compare expressions, and the types
of $A$ and $B$ are equivalent (\cxxref{temp.type}).



\end{comment}

\end{addedblock}
\end{quote}
