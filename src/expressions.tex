

\setcounter{chapter}{4}
\rSec0[expr]{Expressions}

%%
%% Primary Expressions
%%
\rSec1[expr.prim]{Primary expressions}

In this section, add the
\grammarterm{requires-expression} to the rule for 
\grammarterm{primary-expression}.

\begin{quote}
\begin{bnf}
\nontermdef{primary-expression}\br
    \added{requires-expression}
\end{bnf}
\end{quote}

%%
%% Lambda expressions
%%
\rSec2[expr.prim.lambda]{Lambda expressions}

Insert the following paragraph after paragraph 4 to define the
term ``generic lambda''.

\setcounter{Paras}{4}
\pnum
A \defn{generic lambda} is a 
\grammarterm{lambda-expression} where either the
\tcode{auto} \grammarterm{type-specifier} 
(\ref{dcl.spec.auto})
or a
\grammarterm{constrained-type-specifier} 
(\ref{dcl.spec.constr})
appears in a parameter type of the
\grammarterm{lambda-declarator}.


Modify paragraph 5 so that the meaning of a generic lambda is defined 
in terms of its abbreviated member function call operator.

\setcounter{Paras}{5}
\pnum
The closure type for a non-generic
\grammarterm{lambda-expression} has a public inline 
function call operator (\cxxref{over.call})
whose parameters and return type are described by the 
\grammarterm{lambda-expression}'s 
\grammarterm{parameter-declaration-clause} and 
\grammarterm{trailing-return-type}, respectively. 
% 
\removed{
For a generic lambda, the closure type has a public inline function call
operator member template (\ref{temp.mem}) whose 
\grammarterm{template-parameter-list} consists of 
one invented type \grammarterm{template-parameter} 
for each occurrence of \tcode{auto} in the lambda's 
\grammarterm{parameter-declaration-clause}, in order 
of appearance.
% 
The invented type \grammarterm{template-parameter} is 
a parameter pack if the corresponding 
\grammarterm{parameter-declaration declares} a 
function parameter pack (\ref{dcl.fct}). 
% 
The return type and function parameters of the function call operator 
template are derived from the 
\grammarterm{lambda-expression}'s 
\grammarterm{trailing-return-type} and 
\grammarterm{parameter-declaration-clause} by 
replacing each occurrence of \tcode{auto} in the 
\grammarterm{decl-specifier}s of the 
\grammarterm{parameter-declaration-clause}
with the name of the corresponding invented template-parameter.
}
% 
\added{
For a generic lambda, the function call operator is an abbreviated
member function, whose parameters and return type are derived according
to the rules in \ref{dcl.fct}.
}


Add the following example after those in 
\cxxref{expr.prim.lambda}/5. Note that
the existing examples in the original document are omitted in this
document.

\setcounter{Paras}{5}
\pnum
\begin{addedblock}
\enterexample
\begin{codeblock}
template<typename T> concept bool C = true;

auto gl = [](C& a, C* b) { a = *b }; // OK: denotes a generic lambda

struct Fun {
    auto operator()(C& a, C* b) const { a = *b; }
} fun;
\end{codeblock}
\tcode{C} is a 
\grammarterm{constrained-type-specifier},
signifying that the lambda is generic. The generic lambda \tcode{gl}
and the function object \tcode{fun} have equivalent behavior when 
called with the same arguments.
\exitexample
\end{addedblock}


%%
%% Requires expressions
%%
\rSec2[expr.prim.req]{Requires expressions}


\pnum
A \grammarterm{requires-expression} provides a 
concise way to express syntactic requirements on template arguments.
% 
A syntactic requirement is one that can be checked by name lookup 
(\cxxref{basic.lookup}) or by checking
properties of types and expressions. 
% 
A \grammarterm{requires-expression} defines
a conjunction of constraints (\ref{temp.constr}),
which are introduced by its 
\grammarterm{requirement}s.

\begin{bnf}
\nontermdef{requires-expression}\br
  \terminal{requires} requirement-parameter-list requirement-body

\nontermdef{requirement-parameter-list}\br
  \terminal{(} parameter-declaration-clause\opt \terminal{)}

\nontermdef{requirement-body}\br
  \terminal{{} requirement-list \terminal{}}

\nontermdef{requirement-list}\br
  requirement \terminal{;}\br
  requirement-list requirement

\nontermdef{requirement}\br
  simple-requirement\br
  type-requirement\br
  compound-requirement\br
  nested-requirement
\end{bnf}

\pnum
A \grammarterm{requires-expression} has type \tcode{bool}.

\pnum 
A \grammarterm{requires-expression} evaluates to \tcode{true} when all of its
introduced constraints are satisfied (\ref{temp.constr}), and \tcode{false}
otherwise.

\pnum
A \grammarterm{requires-expression} shall appear
only within a concept definition (\ref{dcl.spec.concept}) 
or a \grammarterm{requires-clause} following a
\grammarterm{template-parameter-list} 
(\ref{temp}),
\grammarterm{init-declarator}
(\ref{dcl.decl}),
\grammarterm{function-definition}
(\ref{dcl.fct.def}), or
\grammarterm{member-declarator}
(\ref{class.mem}).

\enterexample
The most common use of 
\grammarterm{requires-expression}s is to define
syntactic requirements in concepts  
such as the one below:
\begin{codeblock}
template<typename T>
  concept bool R() {
    return requires (T i) {
      typename A<T>;
      {*i} -> const A<T>&;
    };
  }
\end{codeblock}
A \grammarterm{requires-expression} can also be
used in a \grammarterm{requires-clause} templates 
as a way of writing ad hoc constraints on template arguments such as 
the one below:
\begin{codeblock}
template<typename T>
  requires requires (T x) { x + x; }
    T add(T a, T b) { return a + b; }
    \end{codeblock}
The first \tcode{requires} introduces the 
\grammarterm{requires-clause}, and the second
introduces the \grammarterm{requires-expression}.
\exitexample


\pnum
A \grammarterm{requires-expression} may introduce
local parameters using a 
\grammarterm{parameter-declaration-clause}
(\ref{dcl.fct}). 
% 
Each name introduced by a local parameter is in scope from the point
of its declaration until the closing brace of the
\grammarterm{requirement-body}.
% 
These parameters have no linkage, storage, or lifetime; they are only used
as notation for the purpose of defining 
\grammarterm{requirements}. 

\pnum
A local parameter shall not have a default argument.

\pnum
The \grammarterm{requirement-parameter-list} shall
not include an ellipsis.

\pnum
The \grammarterm{requirement-body} defines a
block scope, and is comprised of a sequence of 
\grammarterm{requirement}s.
% 
These \grammarterm{requirement}s may refer to local 
parameters, template parameters, and any other declarations visible from the 
enclosing context. 
% 
Each \grammarterm{requirement} appends one or more atomic constraints
(\ref{temp.constr}) to the conjunction of constraints defined by the
\grammarterm{requires-expressions}.


%%
%% Simple requirements
%%
\rSec3[expr.prim.req.simple]{Simple requirements}

\begin{bnf}
\nontermdef{simple-requirement}\br
  expression
\end{bnf}

\pnum
A \grammarterm{simple-requirement} introduces 
an expression constraint (\ref{temp.constr.atom.expr}) 
for its \grammarterm{expression}.
% 
\enterexample
\begin{codeblock}
template<typename T> concept bool C =
  requires (T a, T b) {
    a + b;  // a simple requirement
  };
\end{codeblock}
\exitexample

\pnum
The substitution of template arguments into the
\grammarterm{expression} of a 
\grammarterm{simple-requirement} may result
in an ill-formed expression. However, the resulting program is not 
ill-formed. 

\pnum
If the expression would always result in a substitution failure, 
the program is ill-formed.
\enterexample
\begin{codeblock}
template<typenamne T> concept bool C =
requires () {
new T[-1];  // error: the expression will never be well-formed
};
\end{codeblock}
\exitexample


%%
%% Type requirements
%%
\rSec3[expr.prim.req.type]{Type requirements}

\begin{bnf}
\nontermdef{type-requirement}\br
  typename-specifier
\end{bnf}

\pnum
A \grammarterm{type-requirement} introduces 
type constraint (\ref{temp.constr.atom.type}) for 
the type named by its 
\grammarterm{typename-specifier}.
% 
\enternote
A type requirement asserts the validity of an associated
type, either as a member type, a class template specialization,
or an alias template. It is not used to specify requirements for
arbitrary \grammarterm{type-specifiers}.
\exitnote
% 
\enterexample
\begin{codeblock}
template<typename T> struct S { };
template<typename T> using Ref = T&;

template<typename T> concept bool C =
  requires () {
    typename T::inner; // required nested type name
    template S<T>;   // required class template specialization
    typename Ref<T>; // required alias template substitution
  };
\end{codeblock}
\exitexample


\pnum
The substitution of template arguments into the
\grammarterm{typename-specifier} of a
\grammarterm{type-requirement} may result
in an ill-formed type. However, the resulting program is not 
ill-formed. 

\pnum
If the required type would always result in a substitution failure,
then the program is ill-formed.
\enterexample
\begin{codeblock}
template<typename T> concept bool C =
  requires () {
    typename int::X;  // error: int does not have class type
  };
\end{codeblock}
\exitexample




%%
%% Compound requirements
%%
\rSec3[expr.prim.req.compound]{Compound requirements}
      

\begin{bnf}
\nontermdef{compound-requirement}\br
    \terminal{constexpr}\opt \terminal{{} expression \terminal{}} \terminal{noexcept}\opt trailing-return-type\opt \terminal{;}
\end{bnf}

\pnum
A \grammarterm{compound-requirement} introduces 
a conjunction of one or more atomic constraints related to its
\grammarterm{expression}.
% 
A \grammarterm{compound-requirement} appends
an expression constraint (\ref{temp.constr.atom.expr})
for its \grammarterm{expression},
\tcode{E}.
% 
The \tcode{constexpr} specifier, if present, appends a constant 
expression constraint (\ref{temp.constr.atom.constexpr}) 
for \tcode{E}.
% 
The \tcode{noexcept} specifier, if present, appends an exception 
constraint (\ref{temp.constr.atom.noexcept}) for 
\tcode{E}.
% 
If the \grammarterm{trailing-return-type} is
given, and its type (call it \tcode{T}), contains no placeholders
(\ref{dcl.spec.auto}, \ref{dcl.spec.constr}),
the requirement appends two constraints to the conjunction of 
constraints: 
% 
a type constraint on the formation of \tcode{T}
(\ref{temp.constr.atom.type}) and
% 
an implicit conversion constraint from \tcode{E} to \tcode{T}
(\ref{temp.constr.atom.conv}).
% 
Otherwise, if \tcode{T} contains placeholders, an argument deduction 
constraint (\ref{temp.constr.atom.deduct})
of \tcode{E} against the type \tcode{T}
is appended to the conjunction of constraints. 
% 
In this constraint, deduction is performed against an invented function
template whose declaration is formed according to the rules for
abbreviated functions in \ref{dcl.fct} where 
\tcode{T} is the type of the invented function's only parameter.

\pnum
\enterexample
\begin{codeblock}
template<typename I> concept bool C() { return true; }

template<typename D> concept bool D = s
  requires(T x) {
    {x++};                     // \#1
    {*x} -> typename T::inner; // \#2
    {f(x)} -> const C&;        // \#3
    {g(x)} noexcept;           // \#4
    constexpr {T::value};      // \#5
  };
\end{codeblock}
Each of these requirements introduces a valid expression constraint
on the expression in its enclosing braces.
% 
Requirement \#1 only an expression constraint on \tcode{x++}. It is 
equivalent to a \grammarterm{simple-requirement} 
with the same \grammarterm{expression}.
% 
Requirement \#2 \tcode{*x} introduces three constraints: an 
expression constraint for \tcode{*x}, a type constraint
for \tcode{typename T::inner}, and a conversion constraint
requiring \tcode{*x} to be implicitly convertible to 
\tcode{typename T::inner}.
% 
Requirement \#3 also introduces two constraints: an
expression constraint for \tcode{f(x)}, and a deduction
constraint requiring that overload resolution succeeds for the
call \tcode{i(f(x))} where \tcode{g} is the following
invented function template.
\begin{codeblock}
template<typename T> requires C<T>() void i(const T&);
\end{codeblock}
% 
Requirement \#4 introduces two constraints: an expression constraint
for \tcode{g(x)} and an exception constraint on
\tcode{g(x)}.
% 
Requirement \#5 also introduces two constraints: An expression constraint
on \tcode{T::value} and a constant expression constraint that
\tcode{T::value}.

\pnum
The substitution of template arguments into the
\grammarterm{expression} of a 
\grammarterm{simple-requirement} may result
in an ill-formed expression. However, the resulting program is not 
ill-formed. 


%%
%% Nested requirements
%%
\rSec3[expr.prim.req.nested]{Nested requirements}

\begin{bnf}
\nontermdef{nested-requirement}\br
  requires-clause \terminal{;}
\end{bnf}

\pnum
A \grammarterm{nested-requirement} can be used
to specify additional constraints on a type declared by a
\grammarterm{type-requirement} 
(\ref{expr.prim.req.type}).
% 
It appends a normalization of the 
\grammarterm{constraint-expression}
of its \grammarterm{requires-clause} to the
conjunction of constraints (\ref{temp.constr.expr}).
% 
\enterexample 
\begin{codeblock}
template<typename T> concept bool C() { return sizeof(T) == 1; }

template<typename T> concept bool D =
  requires () {
    typename X<T>;
    requires C<X<T>>();
  };
\end{codeblock}
The \grammarterm{nested-requirement} appends
the predicate constraint \tcode{sizeof(T) == 1}
(\ref{temp.constr.atom.pred}).
\exitexample
