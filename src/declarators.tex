%%
%% Declarators
%%
\rSec0[dcl.decl]{Declarators}

In paragraph 1, modify the grammar of \grammarterm{init-declarator}{} to
allow the specification of constraints on function declarations.

\begin{quote}
\pnum
\begin{bnf}
\nontermdef{init-declarator}\br
    declarator initializer\opt\br
    \added{declarator requires-clause}
\end{bnf}
\end{quote}

Add the following paragraph after paragraph 3.

\setcounter{Paras}{2}
\begin{quote}
\begin{addedblock}
\pnum
The optional \grammarterm{requires-clause} (Clause~\ref{temp}) in an
\grammarterm{init-declarator} or \grammarterm{member-declarator}
shall be present only when the declarator declares a 
function (\ref{dcl.fct}).
% 
When present after a declarator, the \grammarterm{requires-clause} 
is called the \defn{trailing \grammarterm{requires-clause}{}}.
The trailing \grammarterm{requires-clause} introduces the
\grammarterm{constraint-expression} that results from interpreting
its \grammarterm{constraint-logical-or-expression} as a
\grammarterm{constraint-expression}.
% 
\enterexample
\begin{codeblock}
void f1(int a) requires true;         // OK
auto f2(int a) -> bool requires true; // OK
auto f3(int a) requires true -> bool; // error: requires-clause precedes trailing-return-type
void (*pf)() requires true;           // error: constraint on a variable
void g(int (*)() requires true);      // error: constraint on a parameter-declaration
  
auto* p = new void(*)(char) requires true; // error: not a function declaration
\end{codeblock}
\exitexample
\end{addedblock}
\end{quote}


%%
%% Meaning of declarators
%%
\setcounter{section}{2}
\rSec1[dcl.meaning]{Meaning of declarators}


%%
%% Functions
%%
\setcounter{subsection}{4}
\rSec2[dcl.fct]{Functions}

Modify the first part of paragraph 5. The unchanged remainder of the paragraph
is omitted.

\begin{quote}
\setcounter{Paras}{4}
\pnum
A single name can be used for several different functions in a single 
scope; this is function overloading (Clause~\ref{over}). 
%
\removed{All declarations for a function shall agree exactly in both
the return type and the parameter-type-list.}
\added{All declarations for a function shall have equivalent return types, 
parameter-type-lists, and \grammarterm{requires-clause}{s} (\ref{temp.over.link}).}
\end{quote}

Modify paragraph 8 to exclude constraints from the type of a function.
Note that the change occurs in the sentence following the example
in the \Cpp Standard.

\begin{quote}
\setcounter{Paras}{7}
\pnum
The return type, the parameter-type-list, the \grammarterm{ref-qualifier}, 
the \grammarterm{cv-qualifier-seq}, and the exception specification, but not 
the default arguments (\cxxref{dcl.fct.default})
\added{or \grammarterm{requires-clause}{s} (Clause~\ref{temp})}
are part of the function type.
\end{quote}

\setcounter{section}{4}
\rSec1[dcl.fct.def]{Function definitions}

\rSec2[dcl.fct.def.general]{In general}

Change in paragraph 1:

\begin{quote}
\begin{bnf}
\nontermdef{function-definition}\br
  attribute-specifier-seq\opt decl-specifier-seq\opt declarator virt-specifier-seq\opt function-body\br
  \added{attribute-specifier-seq\opt decl-specifier-seq\opt declarator requires-clause function-body}
\end{bnf}
\end{quote}

\rSec2[dcl.fct.def.default]{Explicitly-defaulted functions}

Change in paragraph 1:

\begin{quote}
A function definition \removed{of the form [...]} \added{with the \grammarterm{function-body} \tcode{= default ;}}
is called an \defn{explicitly-defaulted} definition.
\end{quote}

\rSec2[dcl.fct.def.delete]{In general}

Change in paragraph 1:

\begin{quote}
A function definition \removed{of the form [...]} \added{with the \grammarterm{function-body} \tcode{= delete ;}}
is called a \defn{deleted} function.
\end{quote}
