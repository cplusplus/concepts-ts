
%%
%% Declarators
%%
\rSec0[dcl.decl]{Declarators}

Factor the grammar of \grammarterm{declarator}{}s to
allow the specification of constraints on function declarations.

\begin{quote}
\setcounter{Paras}{3}


\begin{bnf}
\nontermdef{declarator}\br
    ptr-declarator\br
    noptr-declarator parameters-and-qualifiers trailing-return-type \added{requires-clause\opt}


\nontermdef{parameters-and-qualifiers}\br
  \terminal{(} parameter-declaration-clause \terminal{)} cv-qualifier-seq\opt\br
    \hspace*{\bnfindentinc}ref-qualifier\opt exception-specification\opt attribute-specifier-seq\opt \added{requires-clause}\opt
\end{bnf}
\end{quote}


Add the following paragraphs at the end of this section.

\begin{quote}
\pnum
The optional \grammarterm{requires-clause} (\ref{temp.constr.decl}) in a 
\grammarterm{declarator} shall be present only when the declarator declares a 
function (\ref{dcl.fct}), and that \grammarterm{requires-clause} shall not 
precede a \grammarterm{trailing-return-type}. 
% 
When present in a \grammarterm{declarator}, the \grammarterm{requires-clause} 
is called the \defn{trailing \grammarterm{requires-clause}{}}.
% 
\enterexample
\begin{codeblock}
void f1(int a) requires true;         // OK
auto f2(int a) -> bool requires true; // OK
auto f3(int a) requires true -> bool; // error: requires-clause precedes trailing-return-type
void (*pf)() requires true;           // error: constraint on a variable
void g(int (*)() requires true);      // error: constraint on a parameter-declaration
  
auto* p = new void(*)(char) requires true; // error: not a function declaration
\end{codeblock}
\exitexample
\end{quote}


%%
%% Meaning of declarators
%%
\setcounter{section}{2}
\rSec1[dcl.meaning]{Meaning of declarators}


%%
%% Functions
%%
\setcounter{subsection}{4}
\rSec2[dcl.fct]{Functions}

Modify the matching condition in paragraph 1 to accept a 
\grammarterm{requires-clause}.
      
\begin{quote}
\pnum
\begin{bnf}
\terminal{D1} \terminal{(} parameter-declaration-clause \terminal{)} cv-qualifier-seq\opt\br
  \hspace*{\bnfindentinc}ref-qualifier\opt exception-specification\opt attribute-specifier-seq\opt\br
  \hspace*{\bnfindentinc}\added{requires-clause\opt}
\end{bnf}
\end{quote}

Modify the matching condition in paragraph 2 to accept a 
\grammarterm{requires-clause}.

\begin{quote}
\pnum
\begin{bnf}
\terminal{D1} \terminal{(} parameter-declaration-clause \terminal{)} cv-qualifier-seq\opt\br 
  \hspace*{\bnfindentinc}ref-qualifier\opt exception-specification\opt attribute-specifier-seq\opt\br
  \hspace*{\bnfindentinc}trailing-return-type \added{requires-clause\opt}
\end{bnf}
\end{quote}


Modify the second sentence of paragraph 5. The remainder of this
paragraph has been omitted.

\begin{quote}
\setcounter{Paras}{4}
\pnum
A single name can be used for several different functions in a single 
scope; this is function overloading (Clause~\ref{over}). 
%
All declarations for a function shall agree exactly in \removed{both} 
the return type\added{,} \removed{and} the parameter-type-list\added{, and 
associated constraints, if any (\ref{temp.constr.decl})}.
\end{quote}

Modify paragraph 6 to exclude constraints from the type of a function.
Note that the change occurs in the sentence following the example
in the \Cpp Standard.

\begin{quote}
\pnum
The return type, the parameter-type-list, the \grammarterm{ref-qualifier}, 
and the \grammarterm{cv-qualifier-seq}, but not the default arguments
(\cxxref{dcl.fct.default})\added{, associated constraints (\ref{temp.constr.decl}),} or
the exception specification (\cxxref{except.spec}), are part of the function 
type.
\end{quote}

Modify paragraph 15. Note that the footnote reference has been
omitted.

\begin{quote}
\setcounter{Paras}{14}
\pnum
There is a syntactic ambiguity when an ellipsis occurs at the end of a 
\grammarterm{parameter-declaration-clause} without a preceding comma. In this 
case, the ellipsis is parsed as part of the \grammarterm{abstract-declarator} 
if the type of the parameter either names a template parameter pack that has 
not been expanded or contains 
\removed{\tcode{auto}}
\added{a placeholder (\ref{dcl.spec.auto})}; 
otherwise, it is parsed as part
of the \grammarterm{parameter-declaration-clause}.
\end{quote}

Add the following paragraphs after paragraph 15.

\begin{quote}
\pnum
An \defn{abbreviated function template} is a function declaration whose
parameter-type-list includes one or more placeholders (\ref{dcl.spec.auto}).
% 
An abbreviated function template is equivalent to a function template
(\ref{temp.fct}) whose \grammarterm{template-parameter-list}
includes one invented \grammarterm{template-parameter} for each occurrence 
of a placeholder in the \grammarterm{parameter-declaration-clause},
in order of appearance, according to the rules below.

\pnum
Each template parameter is invented as follows.
\begin{itemize}
\item If the placeholder is designated by the \tcode{auto}
\grammarterm{type-specifier}, then the corresponding invented template 
parameter is a type \grammarterm{template-parameter}.

\item Otherwise, the placeholder is designated by a 
\grammarterm{constrained-type-specifier}, and the corresponding invented 
parameter is a \grammarterm{constrained-parameter} (\ref{temp.param}) whose 
\grammarterm{nested-name-specifier} and \grammarterm{constrained-type-name} 
are those of the \grammarterm{constrained-type-specifier}.

\item If the placeholder appears in the \grammarterm{decl-specifier-seq} of a 
function parameter pack (\ref{temp.variadic}), or the 
\grammarterm{type-specifier-seq} of a \grammarterm{type-id} that is a pack 
expansion, the invented template parameter is a 
template parameter pack.
\end{itemize}
\enternote
Template parameters are also invented to deduce the type of a variable
whose declared type contains placeholders (\ref{dcl.spec.auto.deduct}).
\exitnote

\pnum 
The adjusted function parameters of an abbreviated function template are derived
from the \grammarterm{parameter-declaration-clause} by replacing each 
occurrence of a placeholder with the name of the corresponding invented 
\grammarterm{template-parameter}.
% 
If the replacement of a placeholder with the name of a template parameter
results in an invalid parameter declaration, the program is ill-formed.
% 
\enternote
Equivalent function template declarations declare the same function template
(\ref{temp.over.link}).
\exitnote
% 
\enterexample
\begin{codeblock}
template<typename T> class Vec { };
template<typename T, typename U> class Pair { };
template<typename... Args> class Tuple { };

void f1(const auto&, auto);
void f2(Vec<auto*>...);
void f3(Tuple<auto...>);
void f4(auto (auto::*)(auto));

template<typename T, typename U> void f1(const T&, U);             // redeclaration of \tcode{f1}
template<typename... T> void f2(Vec<T*>...);                       // redeclaration of \tcode{f2}
template<typename... Ts> void f3(Tuple<Ts...>);                    // redeclaration of \tcode{f3}
template<typename T, typename U, typename V> void f4(T (U::*)(V)); // redeclaration of \tcode{f4}

template<typename T> concept bool C1 = true;
template<typename T> concept bool C2 = true;
template<typename T, typename U> concept bool C3 = true;
template<typename... Ts> concept bool C4 = true;

void g1(const C1*, C2&);
void g2(Vec<C1>&);
void g3(C1&...);
void g4(Vec<C3<int>>);
void g5(C4...);
void g6(Tuple<C4...>);
void g7(C4 p);
void g8(Tuple<C4>);

template<C1 T, C2 U> void g1(const T*, U&); // redeclaration of \tcode{g1}
template<C1 T> void g2(Vec<T>&);            // redeclaration of \tcode{g2}
template<C1... Ts> void g3(Ts&...);         // redeclaration of \tcode{g3}
template<C3<int> T> void g4(Vec<T>);        // redeclaration of \tcode{g4}
template<C4... Ts> void g5(Ts...);          // redeclaration of \tcode{g5}
template<C4... Ts> void g6(Tuple<Ts...>);   // redeclaration of \tcode{g6}
template<C4 T> void g7(T);                  // redeclaration of \tcode{g7}
template<C4 T> void g8(Tuple<T>);           // redeclaration of \tcode{g8}
\end{codeblock}
\exitexample
% 
\enterexample
\begin{codeblock}
template<int N> concept bool Num = true;

void h(Num*); // error: invalid type in parameter declaration
\end{codeblock}
The equivalent declaration would have this form:
\begin{codeblock}
template<int N> void h(N*); // error: invalid type
\end{codeblock}
\exitexample

\pnum
A function template can be an abbreviated function template. The 
invented \grammarterm{template-parameter}{}s are appended to the 
\grammarterm{template-parameter-list} after the explicitly declared 
\grammarterm{template-parameter}{}s.

\enterexample
\begin{codeblock}
template<typename T, int N> class Array { };

template<int N> void f(Array<auto, N>*);
template<int N, typename T> void f(Array<T, N>*); // OK: redeclaration of \tcode{f(Array<auto, N>*)}
\end{codeblock}
\exitexample

\pnum
Two \grammarterm{constrained-type-specifier}{}s are said to be 
\defn{equivalent} if the constraints associated by their corresponding
invented \grammarterm{constrained-parameter}{s} are
equivalent according to the rules for comparing constraints described in 
\ref{temp.constr.constr}.

\pnum
All placeholders introduced by equivalent 
\grammarterm{constrained-type-specifier}{s}
have the same invented template parameter.
% 
\enterexample
\begin{codeblock}
namespace N {
  template<typename T> concept bool C = true;
}
template<typename T> concept bool C = true;
template<typename T, int> concept bool D = true;
template<typename, int = 0> concept bool E = true;

void f0(C a, C b);
\end{codeblock}
The types of \tcode{a} and \tcode{b} are the same invented template
type parameter.
% 
\begin{codeblock}
void f1(C& a, C* b);
\end{codeblock}
The type of \tcode{a} is a reference to an invented template type parameter 
\tcode{T}, and the type of \tcode{b} is a pointer to \tcode{T}.
% 
\begin{codeblock}
void f2(N::C a, C b);
void f3(D<0> a, D<1> b);
\end{codeblock}
In both functions, the parameters \tcode{a} and
\tcode{b} have different invented template type parameters.
% 
\begin{codeblock}
void f4(E a, E<> b, E<0> c);
\end{codeblock}
The types of \tcode{a}, \tcode{b}, and \tcode{c} are the same because the 
\grammarterm{constrained-type-specifier}{s} \tcode{E}, \tcode{E<>}, and 
\tcode{E<0>} all associate the predicate constraint \tcode{E<T, 0>}, where 
\tcode{T} is an invented template type parameter.
% 
\begin{codeblock}
void f5(C head, C... tail);
\end{codeblock}
The types of \tcode{head} and \tcode{tail} are different. Their
respective associated constraints are \tcode{C<T>} and \tcode{C<U>...},
where \tcode{T} is the template parameter invented for \tcode{head} and
\tcode{U} is the template parameter invented for \tcode{U} (\ref{temp.param}).
\exitexample
\end{quote}
